\section{Zusammenfassung}
Ziel der Arbeit ist es, ein Schrittkettensimulationsprogramm unter \ac{codesys} in \ac{fup} zu erstellen und zu simulieren. Hierzu sind im ersten Teil erforderliche Grundlagen dargestellt. Die in der Aufgabenstellung geforderte Einordnung der verwendeten Schichten des OSI-Modells bei einem Regelungssystem wird ebenso erläutert. Im Kapitel \textit{Bohreinrichtung} wird im ersten Schritt anhand der Aufgabenstellung ein Lastenheft erstellt. Darauf aufbauend wird der Programmablauf grafisch in einem Ablaufplan festgehalten. Dieser stellt die Grundlage für die eigentliche Programmierung dar. Zur Überprüfung der Funktionsfähigkeit des Codes wird anschließend eine Simulation der Schrittkette erstellt.\\
Die Grundlagen zu den Sensoren und Aktoren konnte aufgrund des Umfangs des Assignments nur sehr knapp beschrieben werden. Genauso wurde die Entwicklungsumgebung CoDeSys nur in seiner Grundform erläutert. Die gesamten Funktionen, die für die Programmierung erforderlich sind, konnten nicht beschrieben werden. An dieser Stelle ist auf das Benutzerhandbuch \autocite[][]{manCODESYS} zu verweisen. Die Simulation zeigt, dass die Programmierung den geforderten Schrittkettenablauf erfüllt. Das Ziel des Assign\-ments ist somit erreicht.\\
Aufgrund von fehlenden Spezifikationen entsprechend der Maschinenrichtlinien und nicht vorhandenen Sicherheitseinrichtungen für den Maschinenführer wird die Bohreinrichtung jedoch in der Industrie keine Anwendung finden. Dies sollte bei einer möglichen Weiterentwicklung umgesetzt werden. Zusätzlich ist eine Vorschub- und Drehzahlregelung empfehlenswert, um die Bohreinrichtung für verschiedene Werkstoffe und Bohrergrößen nutzen zu können.
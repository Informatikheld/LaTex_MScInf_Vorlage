\section{Einleitung}
\subsection{Problemstellung}
In der Prozessautomatisierung spielt die Durchflussmessung, vor allem in Rohrleitungen, eine wichtige Rolle. Hierfür gibt es verschiedene Messprinzipien. In vielen Industiezweigen hat sich der Vortex-Durchflussmesser als Standard etabliert, welcher das Wirbelfrequenzverfahren nutzt. Verschiedene Hersteller bieten diese Messinstrumente an, jedoch gibt es teilweise Unterschiede in der technischen Umsetzung.\autocites[vgl.][70]{Czichos}[vgl.][519]{Rohrleitung} 
%--------------------------------------------------------------------------------------------------------------------------
%--------------------------------------------------------------------------------------------------------------------------
\subsection{Ziel der Arbeit}
Ziel dieser Arbeit ist die Erarbeitung der Anwendung von Piezo-Sensoren in der Prozessmesstechnik anhand von Wirbelfrequenzzählern. Dem Leser wird ein Einblick in die physikalischen und strömungsmechanischen Grundlagen des Messverfahrens gegeben. Die grundlegende Funktion und das Messprinzip eines Wirbelfrequenzzählers soll ebenso dargestellt werden.\\ 

\subsection{Aufbau der Arbeit}
Die Arbeit beginnt mit den physikalischen Grundlagen. Dabei wird besonders auf den piezoelektrischen Effekt eingegangen. Zusätzlich werden Grundlagen der Strömungsmechanik behandelt. Anschließend kann der Aufbau und das verwendete Messprinzip eines Wirbelfrequenzzählers erläutert werden.\\ Im nächsten Schritt sollen Wirbelfrequenzzähler von zwei verschiedenen Herstellern, die nach dem Vortex-Prinzip arbeiten, vorgestellt werden. Im Anschluss können technische Parameter der beiden Messgeräte gegenübergestellt und verglichen werden. Zum Schluss erfolgt die Zusammenfassung sowie die kritische Auseinandersetzung mit dem erarbeiteten Inhalt.
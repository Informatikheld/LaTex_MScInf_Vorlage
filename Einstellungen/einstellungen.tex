%Einstellugnen für das Dokument, zum Beipiel Formatierung für Verzeichnisse

%Abbildungsverzeichnis
\makeatletter\renewcommand*{\@pnumwidth}{4em}\makeatother
\makeatletter\renewcommand*{\@tocrmarg}{5em}\makeatother
\makeatletter\renewcommand*{\@dotsep }{1}\makeatother

% URL in gleicher Schriftart wie Rest
\urlstyle{same}
% Email mit Verlinkung
\newcommand*{\email}{\href{mailto:lukas.lueck@stud.akad.de}{lukas.lueck@stud.akad.de}} 

%kein einrücken nach Abbildung/Tabelle
\setlength{\parindent}{0pt}

% Formatierungen für das Literaturverzeichnis
%--------------------------------------------
\xpretobibmacro{author}{\begingroup\bfseries}{}{}
\xapptobibmacro{author}{\endgroup}
% Doppelpunkt und Leerzeichen nach Autor
\renewcommand*{\labelnamepunct}{\addspace in\addcolon\addspace} % Punkt am Ende entfernen
\renewcommand*{\finentrypunct}{\addspace}
% Komma zwischen einzelnen Einträgen
\renewcommand*{\newunitpunct}{\addcomma\addspace}
% Punkt am Ende von Zitaten entfernen
\renewcommand*{\bibfootnotewrapper}[1]{\bibsentence#1\addspace}
% Zeilenabstand zwischen einzelnen Einträgen
\setlength{\bibitemsep}{0.5\baselineskip}
%Sortierung bei mehreren Autoren im Format: Name, Vorname
\DeclareNameAlias{sortname}{family-given}
%Trennung zwischen einzelnen Namen mit Semikolon
\renewcommand*{\multinamedelim}{\addsemicolon\space}
\renewcommand*{\finalnamedelim}{\addsemicolon\space}
%Komma zwischen Autor und jahr in Fußnote
\renewcommand*{\nameyeardelim}{\addcomma\space}
%URL ohne Vorgestelltes "URL:"
\DeclareFieldFormat{url}{\addspace\url{#1}}
%number in Runden KLammern
\DeclareFieldFormat{number}{\mkbibparens{#1}}
%URL-Trennung
\apptocmd{\UrlBreaks}{\do\f\do\m}{}{}
\setcounter{biburllcpenalty}{9000}% Kleinbuchstaben
\setcounter{biburlucpenalty}{9000}% Großbuchstaben
\setcounter{biburlnumpenalty}{9000}%Zahlen
%keine Kursive Schrift für Titel im Lit.-Verzeichnis
\DeclareFieldFormat*{title}{#1}


%ABS-WErte mit Skalierten Betragsstrichen
\DeclarePairedDelimiter\abs{\lvert}{\rvert}%
\DeclarePairedDelimiter\norm{\lVert}{\rVert}%

% Swap the definition of \abs* and \norm*, so that \abs
% and \norm resizes the size of the brackets, and the 
% starred version does not.
\makeatletter
\let\oldabs\abs
\def\abs{\@ifstar{\oldabs}{\oldabs*}}
%
\let\oldnorm\norm
\def\norm{\@ifstar{\oldnorm}{\oldnorm*}}
\makeatother

%Farbeinstellung für minted, Matlab
\definecolor{bg}{rgb}{0.95,0.95,0.95}

%Einstellung für pdf-Dokumente, die Übernommen werden, auf Überlappung achten!
%\includepdfset{width=\paperwidth, height=\paperheight, pagecommand={\thispagestyle{fancy}}}

